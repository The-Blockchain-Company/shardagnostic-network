\chapter{Sophie}

\section{Update proposals}
\label{sophie:hardfork}

\subsection{Moment of the hard fork}
\label{sophie:hardfork:moment}

Similar to the Cole ledger (\cref{cole:hardfork:moment}), the Sophie ledger
provides a ``current protocol version'', but it is a two-tuple (not a
three-tuple), containing only a \emph{hard fork} component and \emph{soft fork}
component:
%
\begin{lstlisting}
_protocolVersion :: (Natural, Natural)
\end{lstlisting}
%
in \lstinline!PParams!. The hard fork from Sophie to its successor will be
initiated once the hard fork component of this version gets incremented.

\subsection{The update mechanism for the protocol version}

The update mechanism in Sophie is simpler than it is in Cole. There is no
distinction between votes and proposals: to ``vote'' for a proposal one merely
submits the exact same proposal. There is also no separate endorsement step
(though see \cref{sophie:hardfork:initiating}).

The procedure is as follows:

\begin{enumerate}

\item
As in Cole, a proposal is a partial map from parameters to their values.

\item
During each epoch, a genesis key can submit (via its delegates) zero, one, or
many proposals; each submission overrides the previous one.

\item
``Voting'' (submitting of proposals) ends $6k/f$ slots before the end of the
epoch (i.e., twice the stability period, called \lstinline!stabilityWindow! in
the Sophie ledger implementation).

\item
At the end of an epoch, if the majority of nodes (as determined by the
\lstinline!Quorum! specification constant, which must be greater than half the
nodes) have most recently submitted the same exact proposal, then it is adopted.

\item
The next epoch is always started with a clean slate, proposals from the
previous epoch that didn't make it are discarded.\footnote{Proposals \emph{can}
be explicitly marked to be for future epochs; in that case, these are simply
not considered until that epoch is reached.}

\end{enumerate}

The protocol version itself is also considered to be merely another parameter,
and parameters can change without changing the protocol version, although a
convention could be established that the protocol version must change if any of
the parameters do; but the specification itself does not mandate this.

\subsection{Initiating the hard fork}
\label{sophie:hardfork:initiating}

The timing of the hard fork in Sophie is different to the one in Cole: in
Cole, we \emph{first} vote and then wait for people to get ready
(\cref{cole:hardfork:initiating}); in Sophie it is the other way around.

Core node operators will want to know that a significant majority of the core
nodes is ready (supports the hard fork) before initiating it. To make this
visible, Sophie blocks contain a protocol version. This is not related to the
current protocol version as reported by the ledger state
(\lstinline!_protocolVersion! as discussed in the previous section), but it is
the \emph{maximum} protocol version that the node which produced that block can
support.

Once we see blocks from all or nearly all core nodes with the `hard fork`
component of their protocol version equal to the post-hard-fork value, nodes
will submit their proposals with the required major version change to initiate
the hard fork.\footnote{This also means that unlike in Cole
(\cref{cole:unnecessary-restarts}), in Sophie there is no need to restart the
node merely to support a particular parameter change (such as a maximum block
size).}

\section{Forecasting}
\label{sophie:forecasting}

Discuss the fact that the effective maximum rollback in Sophie is $k - 1$,
not $k$; see also \cref{ledger:forecasting}.
